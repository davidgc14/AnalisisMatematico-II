\documentclass[fleqn]{article}

%\pgfplotsset{compat=1.17}

\usepackage{mathexam}
\usepackage{amsmath}
\usepackage{amsfonts}
\usepackage{graphicx}
\usepackage{systeme}
\usepackage{microtype}
\usepackage{multirow}
\usepackage{pgfplots}
\usepackage{listings}
\usepackage{tikz}
\usepackage{dsfont} %Numeros reales, naturales...
\usepackage{cancel}

%\graphicspath{{images/}}
\newcommand*{\QED}{\hfill\ensuremath{\square}}

%Estructura de ecuaciones
\setlength{\textwidth}{15cm} \setlength{\oddsidemargin}{5mm}
\setlength{\textheight}{23cm} \setlength{\topmargin}{-1cm}



\author{David García Curbelo}
\title{Análisis Matemático II, Prueba 2}
\date{Grado en Matemáticas, Grupo A}

\pagestyle{empty}


\def\R{\mathds{R}}
\def\Z{\mathds{Z}}
\def\N{\mathds{N}}

\def\sup{$^2$}

\def\next{\quad \Rightarrow \quad}

\begin{document}
    \maketitle
    \setcounter{page}{1}
    \pagestyle{plain}

    \textbf{Ejercicio 1.} \\
    
    $$a) \quad \int_1^{\sqrt{3}} \frac{dx}{x^2\sqrt{4-x^2}}$$    

    Para resolver esta integral, usaremos el siguiente cambio de variable, el cual viene dado por 
    \begin{equation*}
        \left[
            \begin{aligned}
                &x = 2\sin(t) \\
                &dx = 2\cos(t) dt
            \end{aligned}
        \right] \\
        \begin{aligned}
            x_2 = \sqrt{3} \Rightarrow 2 \sin(t_2) &= \sqrt{3} \Rightarrow t_2 = \frac{\pi}{3} \\
            x_1 = 1 \Rightarrow 2 \sin(t_1) &= 1 \Rightarrow t_1 = \frac{\pi}{6}
        \end{aligned}
    \end{equation*}
    Y por tanto sustituyendo en la integral de partida obtenemos 
    $$\int_{\frac{\pi}{6}}^{\frac{\pi}{3}} \frac{2\cos(t)}{(2\sin(t))^2 \sqrt{4-( 2\sin(t))^2}}dt = \int_{\frac{\pi}{6}}^{\frac{\pi}{3}} \frac{2\cos(t)}{4\sin^2(t) 2\cos(t)}dt = \frac{1}{4}\int_{\frac{\pi}{6}}^{\frac{\pi}{3}} \frac{1}{\sin^2(t)}dt = $$
    $$ \frac{1}{4}\int_{\frac{\pi}{6}}^{\frac{\pi}{3}} \csc^2(t) dt = \frac{1}{4} \left[-\cot(t)\right]_{\frac{\pi}{6}}^{\frac{\pi}{3}} = \frac{1}{4}\left(-\frac{1}{\sqrt{3}} + \sqrt{3} \right) = \boxed{\frac{1}{2\sqrt{3}}}$$

    $\quad$\\ \\

    $$b) \quad \int_0^{+\infty} \frac{dx}{(x+1)\sqrt{x}}$$    

    Para resolver esta integral, usaremos el siguiente cambio de variable, el cual viene dado por 
    \begin{equation*}
        \left[
            \begin{aligned}
                &x = t^2 \\
                &dx = 2t \thinspace dt
            \end{aligned}
        \right] \\
    \end{equation*}
    Y por tanto sustituyendo en la integral indefinida asociada
    $$\int \frac{2t}{(t^2+1)t} dt = 2 \int \frac{1}{t^2+1} dt = 2\arctan(t) + C$$
    y deshaciendo el cambio de variable obtenemos
    $$\int \frac{dx}{(x+1)\sqrt{x}} = 2\arctan(\sqrt{x}) + C $$
    Por tanto la integral definida resulta
    $$\int_0^{+\infty} \frac{dx}{(x+1)\sqrt{x}} = 2\left[\arctan(\sqrt{x})\right]_0^{+\infty} = 2\left(\frac{\pi}{2} - 0\right) = \boxed{\pi}$$

    \newpage

    \textbf{Ejercicio 2.} \\
   
    $$f(x) = \frac{1}{\sqrt{x}}\sin \left( \frac{1}{x} \right), \quad \forall \thinspace 0 < x < 1$$
    
    Podemos ver que se trata de una función continua definida en un intervalo abierto $]0,1[$, y que estudiando su comportamiento en los extremos del intervalo 
    observamos que
    \begin{equation*}
        \lim_{x \to 0} f(x) = +\infty\\
        \lim_{x \to 1} f(x) = \sin(1)
    \end{equation*}
    Por lo que podemos afirmar que la función es continua en el intervalo $]0,1]$. Veamos si podemos acotar la función por otra función integrable. Para ello 
    tratemos de encontrar una función que acote a nuestra función $f(x)$ y sea integrable. Por ello consideremos la función $g(x) = \frac{1}{\sqrt{x}}$, con la que vemos que
    podemos acotar la función $f(x)$ en el intervalo $]0,1]$, de la siguiente manera
    $$\left| \frac{1}{\sqrt{x}}\sin \left( \frac{1}{x} \right) \right| \leq \frac{1}{\sqrt{x}}$$
    Como sabemos que la función $g(x)$ es una función integrable (por ser de la forma $g(x) = x^a$, con $a>-1$), entonces concluimos que la función $f(x)$ es integrable
    en el intervalo $[0,1]$.\\ \\

    \textbf{Ejercicio 3.} \\

    $$f(x) = e^{\sqrt{x}}, \quad \forall \thinspace 0 \leq x \leq 1$$

    Veamos si se trata de una función integrable. En una primera observación podemos ver que se trata de una función continua en $[0,1]$, donde vemos además que
    \begin{equation*}
        \lim_{x \to 0} f(x) = 1\\
        \lim_{x \to 1} f(x) = e
    \end{equation*}
    Por lo que concluimos que, al tratarse de una función continua y la integral está definida en un intervalo cerrado y acotado, concluimos que la función $f(x)$
    es integrable.

   
\end{document}